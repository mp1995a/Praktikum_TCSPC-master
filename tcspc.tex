% Dokumentenklasse
\documentclass{article}

% deutsche Silbentrennung
\usepackage[ngerman]{babel}

TÄHEST

% Tabellenpaket
\usepackage{tabularx}

% Umlaute
\usepackage[utf8]{inputenc}

% Fließobjekte
\usepackage{float}
% Mathepaket
\usepackage{amsmath}

% Bilder
\usepackage{graphicx}

% PDFs anhängen
\usepackage{pdfpages}

%Hyperref
\usepackage{hyperref}

%Literaturverzeichnis
\usepackage{cite}

%Latexbilder
\usepackage{tikz}


\begin{document}

\begin{titlepage}
\centering
Protokoll zum Fortgeschrittenenpraktikumsversuch\\[2cm]\Huge
\vfill
\Huge
Zeitkorrelierte Einzelphotonenzählung\\[2cm]\Large
WS 2016/2017 \\
\vfill
\normalsize
\begin{tabular}{lcr}
Verfasser: &   & Christoph Egerland \\
Praktikumspartner: &   & Max Pfeifer\\
Versuchsdatum: &   &  6.12.2016 \\
Versuchsplatz: &   & NEW15 2'106\\
Betreuer: &   & Dr. Steffen Hackbarth\\
\end{tabular}
\end{titlepage}

\begin{center}
  \textbf{Abstract}
\end{center}
Mit Hilfe der zeitkorrelierten Einzelphotonenzählung (TCSPC) wird in diesem Versuch die Fluoreszenzlebensdauer von Phäophorbid a in Ethanol-Wasser-Gemischen untersucht.
Hierbei gehen wir insbesondere auf das Verhalten bei unterschiedlichen Konzentrationen, optischen Dichten und
zusätzlichen Reaktionsagenten (Triton X-100) der Probe ein. Es werden Störeffekte (Peak Pile-Up, Reabsorption) behandelt und quantifiziert.
Schließlich wird die Einbettung des Phäophorbid in Mizellen durch Messung der Anisotropie nachgewiesen.

\tableofcontents
\newpage


\section{Materialien und Methoden}
\subsection{Versuchsaufbau}
Der Versuchsaufbau wird in [1] beschrieben:\\
"Der Laser emittiert einen gepulsten Laserstrahl, der auf einen halbdurchlässigen Spiegel trifft.
Dieser teilt den Strahlenweg in zwei Komponenten. Der eine trifft auf eine Referenzdiode. Der
andere wird mit Hilfe eines $\lambda$/2-Plättchens und eines vertikal orientierten Polarisationsfilters
abgeschwächt und trifft anschließend auf die Probe. [...] Bei dem Detektor handelt es sich
um einen Photomultiplier dessen Pulse einen Constant Fraction Discriminator passieren
müssen, um gemessen zu werden."
\begin{figure}[h]
  \centering
  \includegraphics[width=\textwidth]{Bilder/aufbau.jpeg}
  \caption{Versuchsaufbau}
\end{figure}
\\
\\
Die Messelektronik stammt von der Firma Becker und Hickl GmbH, wir verwenden das Programm SPC300 dieser Firma.


\subsection{Versuchsmethode}
Die verwendete Versuchsmethode wird wie folgt in [1] beschrieben: \\
"Mit der TCSPC sollen strahlende optische Übergänge im Bereich vieler ps bis vieler ns
untersucht werden. Nun entsteht aber wegen der Unschärferelation im Allgemeinen das
Problem, nicht gleichzeitig schnell und genau messen zu können. Bei der TCSPC wird dieses
Problem durch die Entkopplung von Detektion und Zeitmessung umgangen. Die Grundidee ist
hierbei, dass sich eine Vielzahl identischer, nicht wechselwirkender Teilchen (oder Moleküle)
statistisch genauso verhält wie ein einzelnes Teilchen (oder Molekül)."

\section{Auswertung und Diskussion}
\subsection{Optimierung des Messplatzes}
In diesem Versuchsteil wollen wir eine optimale Konfiguration (optimal bedeutet hier: höchstmögliches
Signal-Rausch-Verhältnis SNR) erreichen. Wir setzen zunächst eine Streuküvette mit stark verdünntem
Ludox ein, stellen die Detektorspannung auf $U_D=800V$ und regeln die Laserintensität mit dem Polarisationsfilter so niedrig
wie möglich. In Tabelle 1 ist das SNR für die verschiedenen Schwellwerte des Constant-Fraction-Discriminators (CFD) dargestellt.
Wir erreichen das höchste SNR bei: $$CFD_{opt}=30mV$$

\begin{table}[h]
  \centering
  \begin{tabular}{c|c|c|c}
    CFD[mV] & Signal[Counts] & Rauschen[Counts] & SNR \\
    \hline
    5       &     -          & -                & -\\
    10      & 45000          & 200              & 225\\
    15      & 40000          & 50               & 800\\
    20      & 30000          & 10               & 3000\\
    25      & 25000          & 8                & 312.5\\
    30      & 20000          & 6                & 3333.3\\
    35      & 15000          & 5                & 3000\\
    40      & 12000          & 4                & 3000\\
    45      & 8000           & 3                & 2666.7\\
    50      & 5200           & 2                & 2600\\
  \end{tabular}
  \caption{Ermittlung des optimalen Schwellenwerts CFD bei $U=800V$}
\end{table}

Nun variieren wir mit festem Schwellwert $CFD_{opt}$ die Detektorspannung. In Tabelle 2 sehen wir analog zu
Tabelle 1 die Detektorspannung und das daraus resultierende SNR. Wir finden: $$ U_{opt}=800V$$


\begin{table}[h]
  \centering
  \begin{tabular}{c|c|c|c}
    U[kV] & Signal[Counts] & Rauschen[Counts] & SNR \\
    \hline
    0.6   &     0          & 0                & -\\
    0.7   &    520         & 2                & 260\\
    0.8   &  16000         & 5                & 3200\\
    0.9   &     25000      & 8                & 3125\\
    1.0   &     10000      & 5                & 2000\\
  \end{tabular}
  \caption{Ermittlung der optimalen Spannung bei $CFD_{opt}=30mV$}
\end{table}

In den folgenden Versuchsteilen werden wir also stets die feste Konfiguration $CFD_{opt}=30mV$ und $ U_{opt}=800V$ verwenden!

\subsubsection{Apparatefunktion}
Die aufgenommen Apparatefunktion der Streuküvette ist in Abbildung 2 dargestellt. Durch variieren der Countrate ergaben sich
kleine Variationen in der Halbwertsbreite des Peaks, sowie veränderte Fluktuationen in der auslaufenden Kurve. Es wurde jene Countrate
gewählt, bei der eine weitere Verringung die Halbwertsbreite nicht weiter verringert. Am Graphen ist diese gut zu sehen:
$$Countrate \approx \mathcal{O}(10^4)$$

\begin{figure}[h]
  \centering
  \includegraphics[width=\textwidth]{Bilder/apparatekurve.jpg}
  \caption{Apparatekurve}
\end{figure}



\subsubsection{Peak-Pile-Up-Effekt}
Unter dem Peak-Pile-Up-Effekt versteht man, dass zwei Photonen nicht als zwei separate Events registriert werden. Wir erwarten,
dass bei höherer Laserintensität zum einen die Amplitude $A$ größer wird und dass die Fluoreszenzlebensdauer $t$ abnimmt, da es bei
höheren Intensitäten wahrscheinlicher ist, dass zwei Photonen als ein Ereignis gezählt werden und somit nicht doppelt zur Kurve beitragen.
Die Messung wurde mit einer Probe mit $OD = 0.1$ und verschiedenen Laserintensitäten durchgeführt.

\begin{table}[h]
  \centering
  \begin{tabular}{c|c|c|c}
    I[\mu W]      & A[a.u.]  & t[ns]             & \chi^2\\
    \hline
    15.9          &  0.19    & 5.44              & 1.42  \\
    77            &  0.19    & 5.42              & 1.42  \\
    79            &  1.06    & 5.38              & 2.06  \\
    91            &  1.24    & 5.44              & 2.24  \\
  \end{tabular}
  \caption{Messung des Peak-Pile-Up-Effekts}
\end{table}


Die Ergebnisse der Messung sind in Tabelle 3 aufgeführt. Die Fits zeigen eine gute Übereinstimmung zum Modell ($\chi^2-Werte < 2.3$)
und die oben beschriebenen Erwartungen werden bestätigt. Die Halbwertszeit nimmt mit zunehmender Intensität ab, die Amplitude steigt.
Allerdings liegt die Schwankung lediglich in der Fehlerordnung und ist somit also zu vernachlässigen. Wir wählen $I = 77 \mu W$ als
Laserintensität für die folgenden Versuche.



\subsubsection{Reabsorption}
Für diesen Versuchsteil mussten zunächst 5 Proben mit verschiedener optischer Dichte im Bereich
$0.1$ - $1.5$ hergestellt werden. Hierfür wurde auf der Grundlage, dass die optische Dichte eines Stoffes
proportional zur Konzentration desselben ist, folgende Formel hergeleitet:
\begin{equation}
  \omega = \frac{V_{Pheo}}{V_{Rest}}=\frac{OD_{Probe}}{OD_{Pheo}-OD_{Probe}}
\end{equation}

Hierbei ist $OD_x$ die optische Dichte des jeweiligen Stoffes. Für ein Probenvolumen von $2ml$ und eine
optische Dichte des Pheophorbid a von $OD_{Pheo}=2.4$ ergeben sich die Werte wie in Tabelle ZAHL.

\begin{table}[h]
  \centering
  \begin{tabular}{c|c|c|c}
    $OD_{Probe}$ & \omega & $V_{Pheo}$[\mu l] & $V_{Ethanol}$[\mu l] \\
    \hline
    0.1          &  1:23  & 87                & 1913 \\
    0.3          &  1:7   & 250               & 1750 \\
    0.7          &  7:17  & 583               & 1417 \\
    1.1          &  11:13 & 917               & 1083 \\
    1.5          &  5:3   & 1250              & 750  \\
  \end{tabular}
  \caption{Mischung der Proben für verschiedene optische Dichten}
\end{table}



Unter Reabsorption versteht man, dass ein absorbiertes und wieder abgestrahltes Photon erneut absorbiert und emmitiert wird.
Wir erwarten also, da bei höheren optischer Dichte die Wahrscheinlichkeit für einen solchen Fall höher ist, dass die Fluoreszenzlebensdauer
bei höheren optischer Dichte sinkt, da nun ein Photon zwei Counts verursacht . Die Ergebnisse der Messung befinden sich in Tabelle 5.


\begin{table}[h]
  \centering
  \begin{tabular}{c|c|c|c}
    OD           & A[a.u.]& t[ns]             & \chi^2\\
    \hline
    0.1          &  0.01  & 5.91              & -  \\
    0.3          &  0.54  & 5.44              & 1.66  \\
    0.7          &  0.59  & 5.54              & 1.64  \\
    1.1          &  0.63  & 5.62              & 1.79  \\
    1.5          &  0.27  & 5.86              & 1.50   \\
  \end{tabular}
  \caption{Messung Reabsorptionseffekt}
\end{table}


Die Fits zeigen wieder gute Übereinstimmung mit dem Modell ($\chi^2-Werte <2$) und unsere Erwartungen bestätigen sich deutlich.
Die Fluoreszenzlebensdauer steigt bei höheren Dichte um bis zu 8\% an. Im Folgenden werden wir also $OD = 0.2$ wählen.




\subsection{Fluoreszenzlebensdauer von Pheo}
Nun wird die FLuoreszenzlebensdauer bei den optimalen Einstellung bestimmt. Diese sind noch einmal zusammengefasst:

\begin{itemize}
  \item CFD-Spannung: $CFD_{opt}=30mV$
  \item Detektorspannung: $U_D=800V$
  \item Laserintensität: $I=77\mu W$
  \item optische Dichte der Probe: $OD=0.2$
\end{itemize}

Die Fluoreszenzlebensdauer ist: $$t=(5.9 \pm 0.1)ns$$
Diese stimmt sehr gut mit dem Referenzwert in [2] überein (ebenfalls $t=(5.9 \pm 0.1)ns$). Auch der $\chi^2$-Wert weißt auf
eine sehr gute Übereinstimmung mit dem Modell hin.

\begin{figure}[h]
  \centering
  \includegraphics[width=\textwidth]{Bilder/messung_opt.jpg}
  \caption{Mess- und Fitkurve für optimale Einstellung, Fitparameter: $t=5.91ns$, $A=0.001[a.u.]$, $\chi^2=1.24$}
\end{figure}


\subsection{Pheo in versch. Ethanol-Wasser-Gemischen}
Für diesen Versuchsteil stellen wir 7 Proben mit verschiedenem Ethanol-Wasser-Verhältnis her.
Wir setzen in unserer $2ml$ Probe $V_{Pheo}=1ml$, so dass $V_{Rest}=1ml$. Mit $\omega= \frac{V_{Ethanol}}{V_{Rest}}$ erhalten wir:

\begin{table}[h]
  \centering
  \begin{tabular}{c|c|c}
    \omega & $V_{H_2O}$[\mu l] & $V_{Ethanol}$[\mu l] \\
    \hline
     5\%   & 950               & 50                   \\
     15\%  & 850               & 150                  \\
     30\%  & 700               & 300                  \\
     45\%  & 550               & 450                  \\
     60\%  & 400               & 600                  \\
     75\%  & 250               & 750                  \\
     100\% & 0                 & 1000                 \\

  \end{tabular}
  \caption{Mischung für verschiedene Verhältnisse Ethanol:Wasser}
\end{table}

\subsection{Einbettung in Mizellen / Triton X-100}

\subsubsection{Wirkungsweise von Triton X-100}

\subsubsection{Anisotropie}





\section{Schlussfolgerungen}



\section{Anhang}

\subsection{Plots für Reabsorptionseffekt}

\begin{figure}
\begin{tabular}{cc}
  \includegraphics[width=\textwidth/2]{Bilder/FitOD03.jpg}  &   \includegraphics[width=\textwidth/2]{Bilder/FitOD07.jpg}\\
  (a) OD = 0.3, \chi^2=1.66                                 &   (b) OD = 0.7, \chi^2=1.64                               \\
  \includegraphics[width=\textwidth/2]{Bilder/FitOD11.jpg}  &   \includegraphics[width=\textwidth/2]{Bilder/FitOD15.jpg}\\
  (c) OD = 1.1, \chi^2=1.79                                 &   (d) OD = 1.5, \chi^2=1.50                               \\

\end{tabular}
\caption{Fits und Residuen des Reabsorptionseffekt}
\end{figure}


\subsection{Plots für Peak-Pile-Up-Effekt}

\begin{figure}
\begin{tabular}{cc}
  \includegraphics[width=\textwidth/2]{Bilder/Fit1509uW.jpg}  &   \includegraphics[width=\textwidth/2]{Bilder/Fit7700uW.jpg}\\
  (a) I = 15.09 \mu W, \chi^2=1.42                            &   (b) I = 77 \mu W, \chi^2=1.42                             \\
  \includegraphics[width=\textwidth/2]{Bilder/Fit7900uW.jpg}  &   \includegraphics[width=\textwidth/2]{Bilder/Fit9100uW.jpg}\\
  (c) I = 79 \mu W, \chi^2=2.06                               &   (d) I = 91 \mu W, \chi^2=2.44                             \\

\end{tabular}
\caption{Fits und Residuen des Peak-Pile-Up-Effekts}
\end{figure}





\newpage
\section{Literatur}

\begin{itemize}
  \item[ (1) ] Dr. Steffen Hackbarth: "Versuchsskript: Zeitkorrelierte Einzelphotonenmessung"
  \item[ (2) ] Röder et al.: "Photophysical properties of pheophorbide a in solution and in model membrane systems"
\end{itemize}

\end{document}
